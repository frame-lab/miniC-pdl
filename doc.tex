\documentclass{article}

%\usepackage{algorithm}
%\usepackage{algorithmic}
\usepackage[linesnumbered,ruled,vlined]{algorithm2e}
\usepackage{algpseudocode}
\usepackage{enumerate}
\usepackage{float}
\usepackage[utf8x]{inputenc}

%Usar pacote para gramática: grammar

\title{Compiler C to PDL}
\date{09-08-2018}
\author{Philippe Geraldeli Araujo e Allan Patrick}
\newtheorem{theorem}{Theorem}
\newtheorem{definition}{Definição}

\begin{document}

\maketitle
\pagenumbering{gobble}
\newpage
\pagenumbering{arabic}

\section{C to PDL}

\paragraph{Fork do miniC feito por eubnara\\}
Este compilador tem o objetivo de converter C para PDL(Propositional Dynamic Logic). \vspace{10mm}

\noindent \small 
Program := (DeclList)? (FuncList)?   // DeclList FuncList ou DeclList ou FuncList  \\
DeclList := (Declaration)+          // Declaration ou DeclList Declaration \\
FuncList := (Function)+ \\ 
Declaration := Type IdentList \\
IdentList := identifier (, identifier)*  // identifier ou IdentList , identifier \\
Identifier := id ou id [ intnum ]      // (Note) [, ] are not symbols used in regular expression \\
Function := Type id ( (ParamList)? ) CompoundStmt \\
ParamList := Type identifier (, Type identifier)* \\
Type := int ou float \\
CompoundStmt := { (DeclList)? StmtList } \\
StmtList := (Stmt)* \\
Stmt := AssignStmt ou CallStmt ou RetStmt ou WhileStmt ou ForStmt ou IfStmt ou CompoundStmt ou ; \\
AssignStmt :=Assign  \\
Assign := id = Expr ou id [ Expr ] = Expr \\
CallStmt := Call ; \\
Call := id ( (ArgList)? ) \\
RetStmt := return (Expr)? ;  \\
Expr := MINUS Expr $|$ MathRel Eqltop Expr $|$ MathRel $|$ Call $|$ Ids \\
MathRel := MathEql Relaop MathRel $|$ MathEql \\
MathEql := TERM Addiop MathEql $|$ TERM \\
TERM := FACTOR Multop TERM $|$ FACTOR \\
FACTOR := '(' Expr ')' $|$ FLOATNUM $|$ INTNUM \\
Id := ID $|$ ID [ Expr ] \\



So, Our miniC program doesn't follow the rule below.  \\
    1. ++, --  \\
    2. According to this rule  :=  CompoundStmt := { (DeclList)? StmtList }  \\

\section{Algorithm Converter} 
Input = Arquivo em C \\
Output = Arvore/Arquivo
\vspace{10mm}

\begin{algorithm}

\caption{BuildTree(Program* head)}

\If{headDeclaration != NULL}{
	visitDeclaration(headDeclaration)\;
}
\If{headFunction != NULL}{
	visitFunction(headFunction)\;
}

\end{algorithm}

\BlankLine

\begin{algorithm}
\caption{visitDeclaration(DECLARATION* decl)}

\If{DeclList}{
	\If{Declaration}{
		 insert(declarationType)\;
		 insert(declarationId)\;
		 
	}
	\If{DeclList}{
		visitDeclaration(previousDeclaration)\;
	}
}




\end{algorithm}
%\vspace{-10mm}
\begin{algorithm}

\caption{visitFunction(FUNCTION* func)}

\If{FunctionList}{
	\If{FunctionList}{
		visitFunction(previousFunction)\;
	}
	\If{Function}{
		 insert('(')\;
		 \If{funcParameter != NULL}{
		 	insert(funcParameter)\;
		 }
		 visitCompoundStmt(FunctionCstmt)\;
	}
}




\end{algorithm}

\begin{algorithm}

\caption{visitCompoundStmt(COMPOUNDSTMT* cstmt)}

\If{cstmtDeclaration != NULL}{
	visitDeclaration{cstmtDeclaration}\;
}
\If{cstmtStatement != NULL}{
	visitStmt(cstmtStatement)\;
}

\end{algorithm}

\begin{algorithm}

\caption{visitStmt(Stmt* stmt)}

\Switch{stmtS}{
	\Case{Assign}{
		InsertSemicolon()\;
		insert(stmtS\_AssignID)\;
		insert("=")\;
		insert(stmtS\_AssignExpression)\;
	}
	\Case{Call}{
		insertSemicolon()\;
		insert(stmtSCallIdentifier)\;
		insert(CallArg)\;
	}
	\Case{Return}{
		\eIf{Stmt\_Return == NULL}{
			insert("return")\;
		}{
			insert("return")\;
			visitStmt(stmtS\_Return)\;
		}
	}
	\Case{While}{
		\eIf{StmtSdo\_while == true}{
			visitStmt(stmtS\_while)\;
			insert(WhileCondition)\;
			visitStmt(stmtS\_while)\;
			insert(")*;$\neg$(")\;
		}{
			insert(WhileCondition)\;
			visitStmt(stmtS\_while)\;
			insert(")*$\neg$")\;
			insert(WhileCondition)\;
		}
	}
	\Case{For}{
		visitStmtS(StmtSAssign)\;
		insert(ForCondition)\;
		visitStmt(stmtS\_For)\;
		visitStmt(stmtS\_ForInc)\;
		insert(ForCondition)\;

	}
	\Case{If}{
		insert(If\_condition)\;
		VisitStmt(stmtS\_if)\;
		\If{stmtSelse != NULL}{
			insert(IfCondition)\;
			%visitExpr(stmtScondition)\;
			visitStmt(stmtSelse)\;
		}
	}
	\Case{CompoundStmt}{
		visitCompoundStmt{stmtS}\;
	}
	\Case{Semicolon}{
		insert("$;$")\;
	}
}
\end{algorithm}

%\vspace{-50mm}

\section{Prova de corretude do algoritmo:}

Em PDL temos duas estruturas importantes, sendo elas programas e fórmulas, programas atômicos são os menores programas possíveis e formulas atômicas são as menores formulas possíveis. Nesse algoritmo um programa atômico tem no máximo 1 operando, foi escolhido essa forma por convenção de 3 registradores. Sendo assim, toda declaração de variável e função de atribuição com apenas 1 operador, um programa atômico, logo uma lista de declarações é uma lista de programas atômicos. Por exemplo:\\
\BlankLine
int i $\rightarrow$ [int i] \\
\BlankLine
Int i = 10 $\rightarrow$ declaração: [int i] $|$ Função: [i = 10] \\
\BlankLine
Um programa pode ser descrito como uma lista de declarações ou uma lista de funções, sendo que pela recursão toda concatenação dos mesmos é tratada e as declarações são tratadas de forma vista anteriormente. \\
\BlankLine
No caso de termos uma função vai recair em casos mais específicos que vão ser tratados isoladamente. Como por construção a recursão cuida de tratar o caso mais interno para o mais externo recairemos sempre sobre o caso mais básico. Temos então os casos das estruturas de repetição, condição e escopo. A condição de escopo é a mais simples, pois como em PDL é ignorado, um escopo é estruturado toda vez que a recursão entra em uma função, logo esse caso é ignorado pela conversão. Logo mostraremos que para um caso genérico Sigma de um subconjunto de C todas as transformações feitas são válidas já que temos prova de que as transformações de uma linguagem de programação para PDL está correto comprovadamente, logo iremos mostrar que a nossa conversão de algoritmo equivale a conversão que já está comprovada.\\
\BlankLine
Dado pela definição de PDL no livro [1] o Assign é dado pela seguinte forma:\\
\BlankLine
int i = 1 + 2 + 3 $\rightarrow$ declaração: [int i] $|$ função: [ i = 1 + 2 + 3]\\
\BlankLine
Como visto anteriormente toda função Assign utiliza apenas 1 operando para o nosso algoritmo, logo no caso de mais operandos é utilizado Identifiers auxiliares para esse quesito, por exemplo:\\
\BlankLine
int j = 1 $\rightarrow$ declaração: [int j] $|$ função: [ j = 1 ] \\
\BlankLine

\begin{algorithm}
	\caption{void Parts(struct EXPR* expr)}
	
	char var[] = "\_tX"\;
	int buffersize = 100\;
	char* variable = malloc(buffersize)\;
	
	\Switch{exprE}{
		\Case{eId}{
			var[2] = ++x + '0'\;
			strncpy(variable,var,buffersize)\;
			Parts3[tam][0] = variable\;
			tam++\;
		}
		\Case{eIntnum}{
			var[2] = ++x + '0'\;
			strncpy(variable,var,buffersize)\;
			Parts3[tam][0] = variable\;
			tam++\;
		}
		\Case{eFloatnum}{
			var[2] = ++x + '0'\;
			strncpy(variable,var,buffersize)\;
			Parts3[tam][0] = variable\;
			tam++\;
		}		
	}
\end{algorithm}

	Na função Parts temos que:
	\BlankLine
	Na linha 1 é criado o nome da variável. 
	Na linha 2,3  é criado um buffer de tamanho 100, para alocar todas as variáveis necessárias.
	Na linha 8,13,18 é adicionado na tabela o nome da variável.
	\BlankLine
	Logo temos que para o caso simples de 1 operando é trivial verificar que as transformações são idênticas, pois é fácil ver que é uma adição direta na arvore. Para n > 1 operandos temos que é fácil ver ficar também visto que para uma função genérica teremos:
	\BlankLine
	int g = n + (n+1) + (n+2) + (n+3) + ... + (n+m)
	\BlankLine
	
logo a transformação da declaração ficará: 
	\BlankLine
	aux n   = ( n + (m-1) ) + (n+m)  
	aux n-1 = ( n + (m-2) ) + aux n 
	aux n-2 = ( n + (m-3) ) + aux n-1 
	... 
	aux = (n+1) + aux 2 
	\BlankLine
	g = (n) + aux 
	\BlankLine
	O compilador faz essa conversão na função visitExpr2, tendo então que esta transformação é trivialmente transformada em: 
	\BlankLine
	int g = n + (n+1) + (n+2) + (n+3) + ... + (n+m)
	\BlankLine
	Logo como a conversão é equivalente a definição está correto.
	\BlankLine


	\subsection{Caso do IF}
		A conversão para PDL de um if simples ocorre da seguinte forma:
			\begin{definition}[Condicional]
		if a then b else c $\rightarrow$ $a?;b\cup\neg a?;c$
			\end{definition}
	
	Temos que o compilador faz essa conversão da seguinte forma:
	
	\begin{algorithm}[hbt]
		\caption{(void visitIf\_s2(struct IF\_S* if\_s))}
		scopeTail = newScope(sIF, scopeTail)\;
		\_isTitlePrinted2 = false\;
		scopeTail$\rightarrow$parent$\rightarrow$if\_n++\;
		
		insert(aux, "child", "( (")\;
		aux2 = aux\;
		visitExpr2(if\_s$\rightarrow$cond)\;
		insert(aux,"child",")? ")\;
		\If{if\_s$\rightarrow$if\_$\rightarrow$s}{
			insert(aux,"child",";")\;
		}
		visitStmt2(if\_s$\rightarrow$if\_)\;
		insert(aux,"child"," )")\;
		\If{if\_s$\rightarrow$else\_ != NULL}{
			insert(aux, "child", "U($\neg$(")\;
			aux2 = aux\;
			visitExpr2(if\_s$\rightarrow$cond)\;
			insert(aux,"child",")? ")\;
			\If{if\_s$\rightarrow$else\_$\rightarrow$s}{
				insert(aux,"child",";")\;
			}
			visitStmt2(if\_s$\rightarrow$else\_)\;
			insert(aux, "child", ")")\;
		}
	\end{algorithm}
\begin{enumerate}[(i)]
\item	Na linha 4 adiciona dois escopos de parêntese na arvore.
\item	Na linha 6 procura a condição do if para adicionar em seguida no visitExpr2.
\item	Na linha 7 fecha o parêntese e adiciona o “?”.
\item	Na linha 10 vai para o Stmt do if.
\item	Na linha 13 é checado se existe um else, caso exista é adicionado o “U($\neg$(“ na linha 12.
\item	Na linha 17 é adicionado a condição novamente.
\item	Na linha 18 termina o escopo do if.
\end{enumerate}
	
	Logo podemos inferir que os 2 casos de if são trivialmente equivalentes aos casos que ocorrem na definição de PDL.O algoritmo faz essa conversão e insere na árvore os casos possíveis gerando ramificações. Assim levando ao caso base de cada ramificação e sendo tratada pela recursão.
	
	\subsection{Caso Call}
		Nesse caso o que estamos fazendo na prática é nada mais nada menos que entrando em uma função normalmente, logo quando temos a chamada de uma função externa é fácil ver que é apenas uma função que será inserida na árvore normalmente.
	
	\subsection{Caso While e doWhile}
		O * no PDL é representado da forma que pode ser repetido zero ou mais vezes, logo não temos controle por quantas iterações que serão executadas no programa, no while temos essa abstração.
		
		Temos que a conversão de um while para PDL ocorre da seguinte forma de acordo com a definição:
		
			\begin{theorem}
			While a do b $\rightarrow$ (a?;b)*;$\neg$a?
			\end{theorem}
		
		Temos então o algoritmo do while:
		
		\begin{algorithm}
			\eIf{while\_s$\rightarrow$do\_while == true}{
			insert(aux, "child", "(")\;
			visitStmt2(while\_s$\rightarrow$stmt)\;
			insert(aux, "child", "(")\;
			aux2 = aux\;
			visitExpr2(while\_s$\rightarrow$cond)\;
			insert(aux,"child",")?")\;
			insert(aux, "child", ")*;$\neg$(")\;
			aux2 = aux\;
			visitExpr2(while\_s$\rightarrow$cond)\;
			insert(aux,"child",")? ")\;
			}{
			insert(aux, "child", "( (")\;
			aux2 = aux\;
			visitExpr2(while\_s$\rightarrow$cond)\;
			insert(aux,"child",")?")\;
			visitStmt2(while\_s$\rightarrow$stmt)\;
			insert(aux, "child", ")*;$\neg$(")\;
			aux2 = aux\;
			visitExpr2(while\_s$\rightarrow$cond)\;
			insert(aux,"child",")?")\;
			}
		\end{algorithm}
	
		Nas linhas 1,12 é feito um teste caso seja um dowhile ou um while, no caso de ser um do while ele irá continuar na linha 2, caso contrário na linha 12.
		Na linha 2 é inserida um parêntese.
		Na linha 3 é adicionado uma vez tudo que possui dentro do while.
		Na linha 4 é feito mais uma adição de parêntese.
		Na linha 6 é adicionado a condição do dowhile.
		Na linha 8 é tratado da forma de um while normal.
		Na linha 13 é adicionado 2 parênteses.
		Na linha 15 é adicionado a condição do while.
		Na linha 16 se fecha o parênteses e adiciona o “?”.
		Na linha 17 é adicionado tudo que o while possui internamente.
		Na linha 18,20,21 é adicionado a negação da condição e adicionado um ponto de interrogação.
		
		\section{Caso For}
		
		O for também pode ser tratada da mesma forma que o while mas com algumas ressalvas.Logo temos que a conversão de for para PDL ocorre da seguinte forma, no for temos 3 campos, o de declaração, condição e incremento. Sendo d a declaração, c a condição e i o incremento, temos o seguinte exemplo:
		
		\begin{algorithm}
			\For{c}{
				E\;
			}
		\end{algorithm}
		
		Assim a conversão é feita da seguinte forma:
			
		(d;c?;E;i)*$\neg$c?
		
		No algoritmo temos que:
		
		\begin{algorithm}
			\caption{visitFor\_s2 (struct FOR\_S* for\_s)}
			insert(aux, "child","(")\;
			visitAssignStmt2(for\_s$\rightarrow$init)\;
			insert(aux,"child",";(")\;
			aux2 = aux\;
			visitExpr2(for\_s$\rightarrow$cond)\;
			insert(aux,"child",")?")\;
			insert(aux, "child", "; ")\;
			visitStmt2(for\_s$\rightarrow$stmt)\;
			insert(aux,"child", "; ")\;
			visitAssignStmt2(for\_s$\rightarrow$inc)\;
			insert(aux,"child",")*;$\neg$(")\;
			aux2 = aux\;
			visitExpr2(for\_s$\rightarrow$cond)\;
			insert(aux,"child",")?$\neg$for\_s$\rightarrow$cond")\;
		\end{algorithm}
	
		Na linha 1 é adicionado um parêntese.
		Na linha 2 é adicionado a declaração dentro do for, dado como “for\_s$\rightarrow$init”
		Na linha 3 é adicionado mais um parêntese.
		Na linha 5 é adicionada toda a condição do for.
		Na linha 6 se fecha o parêntese da condição do for.
		Na linha 8 é adicionado tudo que estava dentro do for.
		Na linha 10 é adicionado o campo de incremento.
		
		
		Logo podemos verificar trivialmente que a definição do for no PDL dado em [1] e a conversão do compilador são equivalentes e portanto temos que a corretude do for do compilador está correta visto que a corretude de conversão para o pdl está correta.Por fim como estes elementos englobam todo o escopo do subconjunto de c utilizado pelo algoritmo temos que a corretude do mesmo está correta visto que a corretude de cada elemento está correta
		
		\bibliographystyle{apa}
		\bibliography{referencias}
\end{document}